\documentclass[uplatex]{jsarticle}
\usepackage{subfiles}
\usepackage{array,booktabs}
\usepackage{float}
\usepackage{amsmath}
\usepackage[dvipdfmx]{graphicx}
\usepackage[dvipdfmx,bookmarks=true]{hyperref}
\usepackage{pxjahyper}
\usepackage{ascmac}
\usepackage{siunitx}
\graphicspath{{images/}{../images/}}


\begin{document}


\title{先端人工知能論最終課題レポート}
\author{下村拓}
\date{2018/08/03}
\maketitle


\section{設定した問題の概要}
日本人、韓国人、中国人などはとても似た顔をしている。
しかし、我々日本人は微妙な違いを見分けかなりの精度でこれらの国を見分けることができる。
このタスクを機械が同様の精度で行うことができれば様々な場所で活躍できると考えられる。
例えば、液晶に映る日本語と外国語の表示切り替えを前に立っている外国人の数等で、
時間配分を帰ることが可能であると考えている。
よって、今回は日本人か外国人かの判定をCNNに行わせることとする。


\section{利用した手法の概要と工夫点}
モデルのアーキテクチャとしては基本的にAlexNetのものを用いる。
ただし、$50\times50$の画像を入力するため、次元数は異なる。


\section{利用したデータ}
今回はJリーグ選手の顔を用いた。
理由は、外国人選手が含まれていることと一括で大量のラベル付き顔写真が得られるからである。
Jリーグ公式の選手名鑑から$50\times50$の画像と身長や体重、出身地の情報を取得した。
出身地の情報から日本の都道府県であるかないかで日本人と外国人のラベルをつけた。


\section{得られた結果と評価}


\end{document}
